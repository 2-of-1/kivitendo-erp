\documentclass[twoside]{scrartcl}
\usepackage[frame]{xy}
\usepackage[frenchb]{babel}
\usepackage[latin1]{inputenc}
\usepackage{tabularx}
\setlength{\voffset}{0.5cm}
\setlength{\hoffset}{-2.0cm}
\setlength{\topmargin}{0cm}
\setlength{\headheight}{0.5cm}
\setlength{\headsep}{1cm}
\setlength{\topskip}{0pt}
\setlength{\oddsidemargin}{1.0cm}
\setlength{\evensidemargin}{1.0cm}
\setlength{\textwidth}{19.2cm}
\setlength{\textheight}{24.5cm}
\setlength{\footskip}{1cm}
\setlength{\parindent}{0pt}
\renewcommand{\baselinestretch}{1}
\begin{document}

\newlength{\descrwidth}\setlength{\descrwidth}{10cm}

\newsavebox{\hdr}
\sbox{\hdr}{
  \fontfamily{cmss}\fontsize{10pt}{12pt}\selectfont

  \parbox{\textwidth}{
    \parbox[b]{12cm}{
      <%company%>
      
      <%address%>}\hfill
    \begin{tabular}[b]{rr@{}}
    T�l�phone & <%tel%>\\
    T�l�copieur & <%fax%>
    \end{tabular}

    \rule[1.5ex]{\textwidth}{0.5pt}
  }
}
    
\fontfamily{cmss}\fontshape{n}\selectfont

\markboth{<%company%>\hfill <%invnumber%>}{\usebox{\hdr}}

\pagestyle{myheadings}
%\thispagestyle{empty}     use this with letterhead paper

<%pagebreak 90 27 48%>
\end{tabular*}

  \rule{\textwidth}{2pt}
  
  \hfill
  \begin{tabularx}{7cm}{Xr@{}}
  \textbf{Sous-total} & \textbf{<%sumcarriedforward%>} \\
  \end{tabularx}

\newpage

\markright{<%company%>\hfill <%invnumber%>}

\vspace*{-12pt}

\begin{tabular*}{\textwidth}{@{}lp{\descrwidth}@{\extracolsep\fill}rlrrr@{}}
  \textbf{Num�ro} & \textbf{Description} & \textbf{Qt�} &
    \textbf{Unit�} & \textbf{Prix} & \textbf{Remise} & \textbf{Montant} \\
  & report� de la page <%lastpage%> & & & & & <%sumcarriedforward%> \\
<%end pagebreak%>


\fontfamily{cmss}\fontsize{10pt}{12pt}\selectfont

\vspace*{2cm}

<%name%>

<%street%>

<%zipcode%>

<%city%>

<%country%>

\vspace{3.5cm}

\textbf{F A C T U R E}
\hfill
\begin{tabular}[t]{l@{\hspace{0.3cm}}l}
  \textbf{Date de facturation} & <%invdate%> \\
  \textbf{Num�ro de facture} & <%invnumber%> \\
  \textbf{Num�ro de client} & <%customer_id%>
\end{tabular}

\vspace{1cm}

\begin{tabular*}{\textwidth}{@{}lp{\descrwidth}@{\extracolsep\fill}rlrrr@{}}
  \textbf{Num�ro} & \textbf{Description} & \textbf{Qt�} &
    \textbf{Unit�} & \textbf{Prix} & \textbf{Remise} & \textbf{Montant} \\
<%foreach number%>
  <%number%> & <%description%> & <%qty%> &
    <%unit%> & <%sellprice%> & <%discount%> & <%linetotal%> \\
<%end number%>
\end{tabular*}


\parbox{\textwidth}{
\rule{\textwidth}{2pt}

\vspace{0.2cm}

\hfill
\begin{tabularx}{7cm}{Xr@{}}
  \textbf{Sous-total} & \textbf{<%subtotal%>} \\
<%foreach tax%>
  <%taxdescription%> de <%taxbase%> & <%tax%>\\
<%end tax%>
  \hline
  \textbf{Total} & \textbf{<%total%>}\\
\end{tabularx}

\vspace{0.3cm}

\hfill
  Tous les prix indiqu�s sont en \textbf{<%currency%>}.

\vspace{12pt}

<%if notes%>
  <%notes%>
<%end if%>

}

\vfill
\centerline{\textbf{Merci de faire affaire avec nous!}}

\renewcommand{\thefootnote}{\fnsymbol{footnote}}

\footnotetext[1]{\tiny
Le paiement doit �tre acquitt� au plus tard <%terms%> jours � partir de
la date de facturation. Des int�r�ts seront per�us � raison de 1.5\% par
mois apr�s <%duedate%> jusqu'� ce que le paiement soit complet. Les
�l�ments retourn�s seront sujets � un suppl�ment de remmagasinnage de
10\%. Une autorisation de renvoi doit �tre obtenue au pr�alable aupr�s de
<%company%>. Les frais de transports et d'assurance sur les �l�ments
retourn�s devront �tre couvert par le client de fa�on appropri�e.
<%company%> ne peut �tre tenue responsable des dommages survenus pendant
le transit.}

\end{document}
