%
% Bemerkungen zum Vorlagensatz RB von
% Richardson & Büren GbR, Bonn
%
% Hier wurden einige Ideen aufgegriffen, die in folgendem Vortrag
% erwähnt wurden:
% http://www.lx-office.org/uploads/media/Lx-Office_Anwendertreffen_LaTeX-Druckvorlagen-31.01.2011_01.pdf
%
%
% Aufbau:
%   Die documentclass und alle usepackage-Anweisungen sind in
%   'inheaders.tex' ausgelagert. Diese werden von allen Vorlagen via
%   \input eingebunden.
%
%   Desweiteren sind einige Einstellungen und eigene Befehle, die alle
%   Vorlagen verwenden in 'insetting.tex' untergebracht. Auch diese
%   werden mit \input eingebunden.
%   Da in eingebundenen Dateien die Lx-Office-Variablen nicht aufgelöst
%   werden könnnen, werden die hier verwendeten Variablen in jedem
%   Dokument vorher mit \newcommand neu definiert.
%
% Sprachen:
%   In 'insettings.tex' wird an Hand des herangezogenen
%   Vorlagen-Dateinamens die Sprache unterschieden und eine
%   entsprechende Übersetzungsdatei geladen, die Textbausteine
%   bzw. -Schnipsel enthält. Die Vorlagen verwenden nur diese
%   Schnipsel. Im Moment sind die Vorlagenkürzel DE und EN in
%   Benutzung mit den entsprechenden Übersetzungsdateien 'deutsch.tex'
%   und 'english.tex'.
%
%   Die eigentlichen Vorlagen sind gleich, deshalb sind die Dateien
%   für die Sprachen (z.B. invoice_DE.tex) nur symbolische Links auf
%   die Default-Datei ohne Sprachkürzel (z.B. invoice.tex).
%
%
% Mandanten / Firma:
%   Um gleiche Vorlagen für verschiedene Firmen verwenden zu können,
%   wird je nach dem Wert der Lx-Office-Variablen <%employee_company%>
%   ein Firmenverzeichnis ausgewählt (siehe 'settings.tex') in dem
%   Briefkopf, Identitäten und Währungs-/Kontoeinstellungen hinterlegt
%   sind. Ist keine Firma zugeordnet, so wird das Unterverzeichnis
%   'firma' verwendet.
%
% Identitäten:
%    In jedem Firmen-Unterverzeichnis soll einen Datei 'ident.tex'
%    vorhanden sein, die mit \newcommand Werte für \telefon, \fax,
%    \firma, \strasse, \ort, \ustid, \email und \homepage definiert.
%
% Währungen / Konten:
%    Für jede Währung (siehe 'settings.tex') soll eine Datei vorhanden
%    sein, die das Währungssymbol (\currency) und folgende Angaben für
%    ein Konto in dieser Währung enthält \kontonummer, \bank,
%    \bankleitzahl, \bic und \iban.
%    So kann in den Dokumenten je nach Währung ein anderes Konto
%    angegeben werden.
%
% Briefbogen/Logos:
%    Eine Hintergrundgrafik oder ein Logo kann in Abhängigkeit vom
%    Medium (z.B. nur beim verschicken mit E-Mail) eingebunden
%    werden. Dies ist im Moment auskommentiert.
% 
%    Desweiteren sind (auskommentierte) Beispiele enthalten für eine
%    Grafik als Briefkopf, nur ein Logo, oder ein komplletes DinA4-PDF
%    als Briefpapier.
%
% Fusszeile:
%    Die Tabelle im Fuß verwendet die Angaben aus firma/ident.tex und
%    firma/*_account.tex.
%
%
% Tabellen:
%    Als Tabellenumgebung wird longtable verwendet. Diese Umgebung
%    kann in einer Tabelle umbrechen. Da aber der Umbruch nicht von
%    Lx-Office kontrolliert wird, kann man kein Übertrag mit
%    <%sumcarriedforward%> machen (dazu z.B. tablularx und
%    <%pagebreak ... %> verwenden).
%    Innerhalb des Langtextes <%longdescription%> wird nicht
%    umgebrochen. Falls das gewünscht ist, \\ mit \renewcommand
%    umschreiben (siehe dazu:
%    http://www.lx-office.org/uploads/media/Lx-Office_Anwendertreffen_LaTeX-Druckvorlagen-31.01.2011_01.pdf)
%
% Quickstart (wo kann was angepasst werden?):
%    insettings.tex : Pfad zu Angaben über Mandanten (default: firma)
%                     Logo/Briefpapier
%                     Seitenränder / Geomtry
%                     Aussehen Kopf/Fußzeile
%    firma/*        : Angaben über Mandanten
%    deutsch.tex    : Textschnipsel f. Deutsch
%                     Dafür eine Sprache mit Sprachkürzel DE anlegen
%    english.tex    : Textschnipsel f. Englisch
%                     Dafür eine Sprache mit Sprachkürzel EN anlegen
%
