% credit_note.tex
% Verkauf Gutschrift
% Überarbeitet von Norbert Simon, n.simon@linet-services.de
% Version 2.5 vom 16. November 2011
% Basiert auf der Arbeit von kmk@lilalaser.de / 2007
% Diese Vorlage steht unter der GPL-Lizenz, Version 3
% siehe http://www.gnu.de/licenses/gpl-3.0.html
% ----------
% config: tag-style=$( )$
% ----------

\documentclass[twoside]{scrartcl}
\usepackage{fancyhdr}       % Für den Seitenkopf und -Fuß
\usepackage{ifpdf}          % Erlaubt eine Code-Weiche für PDF, oder DVI Ausgabe
\usepackage{xifthen}        % Allgemeine Code-Weiche
\usepackage{graphicx}       % Fuer die Einbindung der Logo-Graphik
\usepackage{german}         % Deutsche Trenn-Tabelle
\usepackage[utf8]{inputenc} % Umlaute direkt eingeben
\usepackage{textcomp}       % Sonderzeichen
\usepackage{lastpage}       % Fuer die Angabe "Seite 2 von 5"
\usepackage{filecontents}   % Um von latex aus eine Datei schreiben zu koennen
\usepackage{etex}           % Damit Marken verwendet werden koennen
\usepackage{ltxtable}       % Mehrseiten-Tabellen mit variabler Spaltenbreite
\usepackage{booktabs}	    % Striche in Tabellen
\usepackage{numprint}       % Zahlen formatiert ausgeben
\usepackage[$(if myconfig_output_numberformat =~ "1.000,00")$german$(else)$$(if myconfig_output_numberformat =~ "1000,00")$germannosep$(else)$$(if myconfig_output_numberformat =~ "1,000.00")$english$(else)$englishnosep$(end)$$(end)$$(end)$]{zwischensumme}  % Lokales Makro zur Berechnung der Zwischensummen
\usepackage{microtype,relsize} %Feinpositionierung, Sperren von Text
\newcommand*{\sperren}[1]{\normalsize\textls*[200]{#1}} %Sperrung Überrschriften

% ---------- Report-Variablen zur Verwendung in lxbriefkopf.tex ----------
% ----------  Die eigenen Daten ----------
\newcommand{\employeename}{$(employee_name)$}
\newcommand{\employeecompany}{$(employee_company)$}
\newcommand{\employeeaddress}{$(employee_address)$}
\newcommand{\employeetel}{$(employee_tel)$}
\newcommand{\employeefax}{$(employee_fax)$}
\newcommand{\employeeemail}{$(employee_email)$}
\newcommand{\employeecoustid}{$(employee_co_ustid)$}
\newcommand{\employeetaxnumber}{$(employee_taxnumber)$}
\newcommand{\employeetable}{tabelle$(employee_login)$.tex}

% ---------- Eigene Bankverbindung falls nicht im Briefkopf gesetzt ----------
% \newcommand{\companybank}{$(company_bank)$}
% \newcommand{\companybankcode}{$(company_bank_code)$}
% \newcommand{\companyaccountnumber}{$(company_account_number)$}

% ---------- Adressat ----------
\newcommand{\name}{$(name)$}
\newcommand{\departmentone}{$(department_1)$}
\newcommand{\departmenttwo}{$(department_2)$}
\newcommand{\cpgreeting}{$(cp_greeting)$}
\newcommand{\cptitle}{$(cp_title)$}
\newcommand{\cpgivenname}{$(cp_givenname)$}
\newcommand{\cpname}{$(cp_name)$}
\newcommand{\street}{$(street)$}
\newcommand{\country}{$(country)$}
\newcommand{\zipcode}{$(zipcode)$}
\newcommand{\city}{$(city)$}
\newcommand{\phone}{$(customerphone)$}
\newcommand{\fax}{$(customerfax)$}
\newcommand{\lettergreeting}{
	\ifthenelse{\equal{$(cp_gender)$}{f}}
	 {Sehr geehrte Frau $(cp_name)$,}
	{\ifthenelse{\equal{$(cp_gender)$}{m}}
	  {Sehr geehrter Herr $(cp_name)$,}
	  {Sehr geehrte Damen und Herren,}
	}\\[1\baselineskip]
}

% ---------- Rechnungsvariablen ----------
\newcommand{\kundennummer}{$(customernumber)$}
\newcommand{\quonumber}{$(quonumber)$}		% Angebotsnummer
\newcommand{\ordnumber}{$(ordnumber)$}		% Auftragsnummer bei uns
\newcommand{\cusordnumber}{$(cusordnumber)$}	% Auftragsnummer beim Kunden
\newcommand{\invnumber}{$(invnumber)$}		% Rechnungsnummer
\newcommand{\invnumbercreditnote}{$(invnumber_for_credit_note)$} %Rechnungsnummer Gutschrift
\newcommand{\docnumber}{Rechnungsnummer: \invnumber}
\newcommand{\quodate}{$(quodate)$}		% Angebotsdatum
\newcommand{\orddate}{$(orddate)$}		% Auftragsdatum
\newcommand{\reqdate}{$(reqdate)$}		% gewuenschtes Lieferdatum
\newcommand{\deliverydate}{$(deliverydate)$}    % Lieferdatum
\newcommand{\invdate}{$(invdate)$}		% Rechnungsdatum
\newcommand{\terms}{$(terms)$}			% Zahlungsfrist
\newcommand{\duedate}{$(duedate)$}		% Fälligkeitsdatum
\newcommand{\invtotal}{$(invtotal)$}		% Gesamtbetrag
\newcommand{\paid}{$(paid)$}			% Schon bezahlt
\newcommand{\total}{$(total)$}			% Restbetrag

% ---------- Lieferadresse ----------
\newcommand{\shiptoname}{$(shiptoname)$}
\newcommand{\shiptocontact}{$(shiptocontact)$}
\newcommand{\shiptodepartmentone}{$(shiptodepartment_1)$}
\newcommand{\shiptodepartmenttwo}{$(shiptodepartment_2)$}
\newcommand{\shiptostreet}{$(shiptostreet)$}
\newcommand{\shiptocity}{$(shiptocity)$}
\newcommand{\shiptocountry}{$(shiptocountry)$}
\newcommand{\shiptophone}{$(shiptophone)$}
\newcommand{\shiptozipcode}{$(shiptozipcode)$}
\newcommand{\shiptofax}{$(shiptofax)$}

% ---------- Währungszeichen ----------
\newcommand{\currency}{$(currency)$}
\ifthenelse{\equal{\currency}{EUR}}{\let\currency\euro}{}
\ifthenelse{\equal{\currency}{YEN}}{\let\currency\textyen}{}
\ifthenelse{\equal{\currency}{GBP}}{\let\currency\pounds}{}
\ifthenelse{\equal{\currency}{USD}}{\let\currency\$}{}

% ---------- Ende Reportvariablen-Umsetzung ----------

% ---------- Briefkopf dazuladen ----------
\input{lxbriefkopf}

\begin{document}
% ---------- Schrift Hauptdokuments (Computermodern-sanserif)  ----------
% \fontfamily{cmss}\fontsize{10}{12pt plus 0.12pt minus 0.1pt}\selectfont
% ---------- Schrift Helvetica ------------------------
\fontfamily{phv}\fontsize{10}{12pt plus 0.12pt minus 0.1pt}\selectfont

% ---------- Firmenlogo nur erste Seite ----------
\thispagestyle{briefkopf}

% ---------- Datum und Nummern ----------
% Position unterhalb des Briefkopfs
\vspace*{\vlogospacing}
\renewcommand{\arraystretch}{0.9}
\begin{minipage}[b]{177mm}
\sperren{\textbf{Gutschrift Nr. \invnumber}}
\hfill
	\small
	\begin{tabular}[b]{r@{\hspace{2mm}}p{\hlogospacing}}
		\textbf{Seite} & {\thepage} von \pageref{LastPage}\\
  		\textbf{Datum} & \invdate \\
		\textbf{Kunden Nr.} & \kundennummer\\
		\nonemptyline{\textbf{Auftrag Nr.} &}{\ordnumber}
		\nonemptyline{\textbf{Rechnung Nr.} &}{\invnumbercreditnote}
		\nonemptyline{\textbf{Gutschrift Nr.} &}{\invnumber}
		\textbf{Ansprechpartner} & \employeename\\
		\nonemptyline{\textbf{Durchwahl} &}{\employeetel}
		\nonemptyline{\textbf{E-Mail} &}{\employeeemail}
	\end{tabular}\\[10mm plus 20mm minus 10mm]
\end{minipage}
\renewcommand{\arraystretch}{1}
\normalsize
% ---------- Begrüßung und Bemerkungen ----------
\vspace{ 5mm}
%\lettergreeting
Hiermit erstatten wir Ihnen zur Rechnung Nr. \invnumbercreditnote{ } die nachfolgenden Positionen.\\
Für Nachfragen steht Ihnen \employeename \ per Telefon (\employeetel) oder per E-Mail (\employeeemail) gerne zur Verfügung.
%\\[0.4\baselineskip]
\ifthenelse{\isempty{$(notes)$}}{}{
      $(notes)$
      }%
\vspace{1\baselineskip}\\
%Mit freundlichen Grüßen\\[1\baselineskip]
%\employeename\\[1\baselineskip]
% ---------- Die eigentliche-Tabelle ----------
% ---------- Tabelle puffern ----------
\begin{filecontents}{\employeetable}
% ---------- globale Variable laufsumme deklarieren ----------
\resetlaufsumme
% ---------- Spaltendefinition ----------
%\begin{longtable}{@{}rlX@{ }rlrr@{\makebox[\widthof{\textbf{~\currency}}]}}
\begin{longtable}{@{}rlX@{ }rlrr@{\makebox[\widthof{\textbf{}}]}}
% ---------- Kopfzeile der Tabelle ----------
	\textbf{Pos} &
	\textbf{Art.Nr.} &
	\textbf{Bezeichnung} &
	\textbf{Menge} &
	\textbf{ME} &
	\textbf{EP/€} &
	\textbf{GP/€} \\
	\midrule
  \endfirsthead
% ---------- Tabellenkopf nach dem Umbruch ----------
	\textbf{Pos} &
	\textbf{Art.Nr.} &
	\textbf{Bezeichnung} &
	\textbf{Menge} &
	\textbf{ME} &
	\textbf{EP/€} &
	\textbf{GP/€} \\
        \midrule
	& & \multicolumn{4}{r}{} & \MarkUebertrPos\\
  \endhead
% ---------- Fuss der Teiltabellen ----------
	\midrule
	& & \multicolumn{4}{r}{} & \MarkZwsumPos \\
  \endfoot
% ---------- Das Ende der Tabelle ----------
  	\midrule
%	& & \multicolumn{4}{r}{ Nettobetrag:} & \MarkZwsumPos \\
  \endlastfoot
% ---------- Positionen ----------
$(foreach number)$
	$(runningnumber)$ &
	$(number)$ &
	$(description)$
% 	\ifthenelse{\equal{$(longdescription)$}{}}{}{\newline
% 	\renewcommand{\baselinestretch}{1}\footnotesize
% 	{\footnotesize $(longdescription)$
% 	\renewcommand{\baselinestretch}{1}\normalsize
% 	}}
 	\ifthenelse{\equal{$(deliverydate_oe)$}{\leer}}{}{
 	        \newline Lieferdatum:~$(deliverydate_oe)$}
 	&
	$(qty)$ &
	$(unit)$ &
	\ifthenelse{\isempty{$(sellprice)$}}{&}{
		\numprint{$(sellprice)$}
		\ifthenelse{\equal{$(p_discount)$}{0}}{}{ -$(p_discount)$\%} &
		\numprint{$(linetotal)$}\Wert{$(linetotal NOFORMAT)$}  
	}\\ %
  $(end number)$

\end{longtable}
% ----------  Ende der Hilfsdatei ----------
\end{filecontents}
% ---------- Puffertabelle öffnen ----------
\LTXtable{\textwidth}{\employeetable}
%---------- Bereich für die Summen ----------
\parbox{\textwidth}{
%---------- Summenbereich nach recht schieben  ----------
\hfill
\setlength{\tabcolsep}{0mm}
\begin{tabular}{@{}r@{ }r@{ }l}
   {Nettobetrag:}& \numprint{$(subtotal)$}& \currency\\
% ---------- Alle Steuern ausweisen ----------
   $(foreach tax)$
%       {$(taxdescription)$ auf }\numprint{$(taxbase)$}~\currency: & \numprint{$(tax)$}& \\
		{$(taxdescription)$}:  & \numprint{$(tax)$}& \currency\\
   $(end tax)$
   \midrule 
   {\textbf{Rechnungsbetrag:}} & \bfseries\numprint{\invtotal} & \textbf{\currency}\\
% ---------- Wenn bereits etwas gezahlt wurde ----------
$(if invtotal != total)$
   	$(foreach payment)$
   	   abzgl. Zahlung vom {$(paymentdate)$}:& {\numprint{-$(payment)$}} & \currency\\
      	$(end paymentdate)$
      	\midrule
      	\textbf{Verbleibend: } & \textbf{\numprint{\total}} & \textbf{\currency}\\
$(end)$
\bottomrule
 \end{tabular}
} %Ende des Summenkasten
\vfill
% ---------- Nachbemerkung mit max. Abstand nach unten ----------
{
%Soweit nicht anders angegeben,
%\ifthenelse{\equal{\deliverydate}{\leer}}
%   {entspricht das Leistungsdatum dem Rechnungsdatum.}
%   {wurde die Leistung am {\deliverydate} erbracht.}\\[0.5em]
%Bitte überweisen Sie den Rechnungsbetrag in Höhe von
%{\numprint{\total}~\currency} innerhalb von
%\ifthenelse{\equal{\duedate}{\leer}}{{14}}{{\terms}}~Tagen
%auf das unten angegebene Konto.
%\ifthenelse{\equal{\duedate}{\leer}}{}
%  {Nach dem {\duedate} behalten wir uns Verzugszinsen vor.}
Bitte nennen Sie uns eine Bankverbindung auf welche das Guthaben überwiesen werden soll.\\
\vfil
\footnotesize
Bereits gelieferte Waren bleiben bis zur vollständigen Bezahlung der
Rechnung unser Eigentum.
}

\end{document}
