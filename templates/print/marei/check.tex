\documentclass[twoside]{scrartcl}
\usepackage[frame]{xy}
\usepackage{tabularx}
\usepackage[utf8]{inputenc}
\setlength{\voffset}{0.4cm}
\setlength{\hoffset}{-2.0cm}
\setlength{\topmargin}{0cm}
\setlength{\headheight}{0.0cm}
\setlength{\headsep}{1cm}
\setlength{\topskip}{0pt}
\setlength{\oddsidemargin}{1.0cm}
\setlength{\evensidemargin}{1.0cm}
\setlength{\textwidth}{17cm}
\setlength{\textheight}{24.5cm}
\setlength{\footskip}{1cm}
\setlength{\parindent}{0pt}
\renewcommand{\baselinestretch}{1}
\begin{document}


\fontfamily{cmss}\fontsize{9pt}{9pt}\selectfont

\parbox[t]{12cm}{
  <%company%>

  <%address%>}
\hfill
\parbox[t]{6cm}{\hfill <%source%>}

\vspace*{0.6cm}

<%text_amount%> \dotfill <%decimal%>/100 \makebox[0.5cm]{\hfill}

\vspace{0.5cm}

\hfill <%datepaid%> \makebox[2cm]{\hfill} <%amount%>

\vspace{0.5cm}

<%name%>

<%street%>

<%zipcode%>

<%city%>

<%country%>

\vspace{2.8cm}

<%company%>

\vspace{0.5cm}

<%name%> \hfill <%datepaid%> \hfill <%source%>

\vspace{0.5cm}
\begin{tabularx}{\textwidth}{lXrr@{}}
\textbf{Rechnung} & \textbf{Ausgestellt}
  & \textbf{Fällig} & \textbf{Verrechnet} \\
<%foreach invnumber%>
<%invnumber%> & <%invdate%> \dotfill
  & <%due%> & <%paid%> \\
<%end invnumber%>
\end{tabularx}

\vfill

\end{document}

